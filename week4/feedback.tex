\documentclass[a4paper]{article}

% Import some useful packages
\usepackage[margin=0.5in]{geometry} % narrow margins
\usepackage[utf8]{inputenc}
\usepackage[english]{babel}
\usepackage{hyperref}
\usepackage{minted}
\usepackage{amsmath}
\usepackage{xcolor}
\definecolor{LightGray}{gray}{0.95}

\title{Peer-review of assignment 4 for \textit{INF3331-Yrjan}}
\author{Andreas Schafferer, Git-repo INF3331-HaakonAndreas, {haakoasc@student.matnat.uio.no} \\
 		Anders Aase, Git-repo INF3331-Anders, {anders.aase91@gmail.com} \\
		Gorm Lundestad, {gorml@student.matnat.uio.no }}
\date{Deadline: Tuesday, 13. October 2015, 23:59:59.}

\begin{document}
\maketitle



\

\section{Review}\label{sec:review}

The assignment was tested using Python 2.7 on OSX El Capitan.

%%%%%%%%%%%%%%%%%%%%%%%%%%%%%%%%%%%%%%%%%%%%%%%%%%%%%%%%%%
\subsection*{Assignment 4.1: Retrieve web page}
Very nice and short function, does excactly what it's supposed to do.


%%%%%%%%%%%%%%%%%%%%%%%%%%%%%%%%%%%%%%%%%%%%%%%%%%%%%%%%%%
\subsection*{Assignment 4.2: Find link to location}
Very good function, clean and easy to read. I especially like how you concatinate different parts of the regex, instead of having multiple long regexes for stadnamn, kommune and fylke. \newline\newline
The code finds kommune, stadnamn, fylke and wildcards excatly as it's supposed to. Well done!\newline\newline
If i have to suggest some changes, i would consider turning these simple if and else statements into single lines, see below:

\begin{minted}[bgcolor=LightGray, linenos, fontsize=\footnotesize]{python}
#Change from
if location:
        # Escape the user given string and replace wildcards with .*
        search = re.sub("\\\\\*", "(?:\S| )*", re.escape(location))
    else:
        # Wildcard if there's no location string.
        search = r"(?:\S| )*"

#to:
# Escape the user given string and replace wildcards with .*
# Wildcard if there's no location string.
if location: search = re.sub("\\\\\*", "(?:\S| )*", re.escape(location))
else: search = r"(?:\S| )*" 
\end{minted}
\newline
This is of course highly subjective, and up to each programmers preferences, but i personally find it cleaner and simpler to read.

%%%%%%%%%%%%%%%%%%%%%%%%%%%%%%%%%%%%%%%%%%%%%%%%%%%%%%%%%%
\subsection*{Assignment 4.3: Retrieve weather information}
Again, very nice function. I really like your use of named tuples, i did not know that module.
Like above, if i have to suggest changes, it would be shortening the code, see below:

\begin{minted}[bgcolor=LightGray, linenos, fontsize=\footnotesize]{python}
if not name_match:
        sys.stderr.write("Failed to parse data from URL: {}\n".format(url))
        return None
    else:
        name = name_match.group(1)"
#The above else statement is excessive, because of the return-statement in the if-sentence.

if not name_match:
        sys.stderr.write("Failed to parse data from URL: {}\n".format(url))
        return None
name = name_match.group(1)"
\end{minted}
\newline
Again, this is my subjective, personal preference. I'm sure you are already aware of this.

%%%%%%%%%%%%%%%%%%%%%%%%%%%%%%%%%%%%%%%%%%%%%%%%%%%%%%%%%%
\subsection*{Assignment 4.4: Buffer all internet activity}
Real nice buffer! I really like how you save all the files to disk using the url as filename. I'm only a little concerned about the performance when you write to disk every time you fetch a new URL, I think this can slow down the program. Could it be an idea to write a function that saves all the files to disk after you have done a huge number of retrievals?
\newline
The buffer works perfectly, well done!



%%%%%%%%%%%%%%%%%%%%%%%%%%%%%%%%%%%%%%%%%%%%%%%%%%%%%%%%%%
\subsection*{Assignment 4.5: Create weather forecast}
This task is working just as expected. Handles none asci-charcters nicely. Really good documentation, and sensible variable names. I like how you implemented the execution of the program with the \newline
\begin{minted}[bgcolor=LightGray, linenos, fontsize=\footnotesize]{python}
if __name__ == "__main__":
\end{minted}
statement. It's easy to test and use.

%%%%%%%%%%%%%%%%%%%%%%%%%%%%%%%%%%%%%%%%%%%%%%%%%%%%%%%%%%
\subsection*{Assignment 4.6: Testing the code}
All py.test passes, and rigorous tests. All doctests passes.
Real good and rigid solution in "test\_buffer\_timestamp", i'm impressed!


%%%%%%%%%%%%%%%%%%%%%%%%%%%%%%%%%%%%%%%%%%%%%%%%%%%%%%%%%%
\subsection*{Assignment 4.7: Extreme places in Norway}
Works like a charm. Simple and easy to understand.


\subsection{Advice and comments}
Very nice solution, real good and easy to understand, thanks to good use of docstrings and useful comment and descriptive variable names. I'm real impressed by your implementation, it was a joy going reviewing the assignment. \newline
The only improvement i could think of would be changing the buffer, so it doesn't write to disk after each retrieval as mentioned above, but this has no impact on the score. 
\section{Estimated points}\label{sec:points}
Perfect assignment, 30 of 30 points!


\bibliographystyle{plain}
\bibliography{literature}

\end{document}
