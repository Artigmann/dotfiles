\documentclass[]{article}
%\usepackage{amsmath}
%\usepackage{amssymb}
\usepackage[utf8]{inputenc}
%\usepackage{color}


%opening
\title{Report week 8 - INF3331}
\author{Yrjan Skrimstad}

\begin{document}
\maketitle
\section{Implementation}
\subsection{Files}
\begin{itemize}
	\item heat\_equation.py: Contains the basic python solver and the plotting
		function.
	\item heat\_equation\_numpy.py: Contains the numpy solver.
	\item heat\_equation\_weave.py: Contains the weave solver written with use
		of inline C.
	\item heat\_equation\_ui.py: Contains the user interface for the heat
		equation solver.
\end{itemize}
\subsection{Details}
\textbf{heat\_equation.py:}
Contains the basic solver \textbf{heat\_equation()}. This is simply a solver
that iterates through the timesteps and executes the given function for every
single point in the array.

The plotting function \textbf{heat\_equation\_plot()}
is a fairly simple plotting function that takes the equation parameters and
calls the given solver (it defaults to the basic python solver) to solve the
equation and then plots the data.\\

\noindent\textbf{heat\_equation\_numpy.py:}
Fairly simple solver called \textbf{heat\_equation\_numpy()}. The
solver uses vectorization and is therefore quite a lot faster than the
python-version.\\

\noindent\textbf{heat\_equation\_weave.py}
Contains the fastest solver I've written, \\\textbf{heat\_equation\_weave()}.
This solver is written with inline C and pointer swapping to avoid unnecessary
copying of the data. There's several type conversions to make sure the inline
C-code behaves as expected. The C code returns a boolean to tell us which array
is currently the new array, this is necessary because of the pointer swapping.\\

\noindent\textbf{heat\_equation\_ui.py}
Contains a user interface for solving and plotting equations. This is written as
a fairly basic python script without unnecessary functions and other fanciness.
It supports quite a few options, run the script with the \emph{-h} flag to see
them.


\section{Runtime comparison}
\subsection{Test system}
Debian Linux - Python 2.7.10 - Intel i5-2450M CPU @ 2.50GHz - 8GB RAM
\subsection{Timings}
These timings are taken from runs with the example data as given by the
assignment text. \\
\begin{tabular}{| c | c |}
	\hline
	Time in seconds & Solver implementation \\
	\hline
	77.510 & Python \\
	1.082 & NumPy \\
	0.217 & Weave, inline C \\
	\hline
\end{tabular} \\
We here see that the pure Python-implementation is definitely the slowest, it's
actually slower than the inline C-implementation by a factor of almost 400. This
makes it practically unusably slow in a lot of situations.

The NumPy-implementation is quite fast, and actually usably fast. However it's
still a lot slower than the inline C-implementation.

The Weave/inline C-implementation is definitely the fastest. It's a lot faster
than the other implementations and would therefore be recommended for use with
any larger datasets.
\end{document}
